\documentclass[a4paper,11pt]{article}%
\usepackage{a4wide}%

\usepackage[francais]{babel}%
\usepackage[utf8]{inputenc}%
\usepackage[T1]{fontenc}%

\usepackage{graphicx}%
\usepackage{xspace}%

\usepackage{url} \urlstyle{sf}%
\DeclareUrlCommand\email{\urlstyle{sf}}%

\usepackage{mathpazo}%
\let\bfseriesaux=\bfseries%
\renewcommand{\bfseries}{\sffamily\bfseriesaux}

\newenvironment{keywords}%
{\description\item[Mots-clés.]}%
{\enddescription}
\usepackage{tabularx}

\newenvironment{remarque}%
{\description\item[Remarque.]\sl}%
{\enddescription}

\font\manual=manfnt
\newcommand{\dbend}{{\manual\char127}}

\newenvironment{attention}%
{\description\item[\dbend]\sl}%
{\enddescription}

\usepackage{listings}%

\lstset{%
  basicstyle=\sffamily,%
  columns=fullflexible,%
  language=caml,%
  frame=lb,%
  frameround=fftf,%
}%

\lstMakeShortInline{|}

\parskip=0.3\baselineskip%
\sloppy%

%%%%%%%%%%%%%%%%%%%%%%%%%%%%%%%%%%%%%%%%%%%%%%%%%

\begin{document}

\title{Approximation de la codéviance}

\author{Rémi Hutin \and Rémy Sun}

\date{23 Novembre 2015}

\maketitle

\begin{abstract}
  Dans ce rapport, nous rendrons compte de l'implémentation en python de l'algorithme CM-sketch d'approximation de la codéviance, ainsi que de l'analyse statistique obtenue
  en appliquant l'implémentation à un triplet de traces réelles.
\begin{keywords}
  Codéviance; CM-sketch
\end{keywords}
\end{abstract}

\section{Outils de travail}
\label{sec:outils}

\subsection{Pourquoi Python?}

\paragraph{Structure de liste/tableau}La structure de liste en python a la particularité de permettre directement l'accés à n'importe quel élément, ce qui permet une implémentation plus comfortable des vecteurs de fréquence.
Inversement, on pourrait dire que les tableaux pythons ont une taille dynamique, ce qui est utile pour la ``standardisation'' de vecteurs.

\paragraph{Beaucoup de fonctions natives}Python posséde beaucoup de bibliothéques évitant d'avoir à implémenter nous même des fonctions comme le logarithme

\paragraph{Typage dynamique}Nous avons été ammenés à manipuler beaucoup de types différents, le typage python permet d'éviter des notations très lourde, ce qui aurait été le cas avec un langage fortement typé comme Ocaml.

\subsection{Parsage des traces}

\paragraph{}En l'état, les traces fournies sont difficiles à exploiter puisqu'il s'agit de lignes de textes (ce sont des logs). De fait, nous avonc écrit une fonction de parsage\newline
en tirant avantage du fait que python supporte la reconnaissance d'expressions rationnelles. Par exemple, le typage du protocole epahttp est fourni ci-dessous:

\begin{lstlisting}
pattern["epahttp"] = re.compile("""
(?P<host>.*)
\s\[
(?P<D>[0-9]*)
:
(?P<H>[0-9]*)
:
(?P<M>[0-9]*)
:
(?P<S>[0-9]*)
\]\s"
(?P<request>.*)
"\s
(?P<code>[0-9]*)
\s
(?P<size>[0-9]*)
""", re.VERBOSE)
\end{lstlisting}

\paragraph{}A partir de cette définition, on peut poser une fonction parseFile qui crée un vecteur qui découpe chaque ligne en différents blocs correspondant à une information différente. Nous nous sommes interessé dans ce document\newline
à l'adresse hôte (``host'')

\subsection{Fonction de hachage 2-Universelle}

\paragraph{}Il est nécessaire d'avoir une famille de fonctions de hachage 2-universelles. La famille retenue est celle des add-multiply-shift qui permet d'obtenir une fonction de hachage très rapidement\newline
Cette famille est définie dans une classe, de sorte à pouvoir créer une nouvelle instance dès qu'il faut utiliser une nouvelle fonction de hachage randomisée

\begin{lstlisting}
 class Hash:
    def __init__(self, w, M):
        self.w = w
        self.M = M
        self.a = random.randint(1, 1 << (w-1)) - 1
        self.b = random.randint(1, 1 << (w-M)) - 1
        
    def __call__(self, x):
        return (((2*self.a-1)*x+self.b) % (1 << self.w)) // (1 << (self.w-self.M))
\end{lstlisting}

\subsection{Standardisation des flux parsés}

\paragraph{}Nous avonc créé une fonction makestandard qui force transforme les information extraite au moment du parsage (``chaines de caractéres'') en des données numériques exploitables.\newline
A chaque chaine de caractére de la liste parsée, on associe un numéro, de sorte à pourvoir traiter le probléme On arrete aussi le parsage au 25000 premières lignes.

\begin{attention}
 Si deux lignes possédent les mêmes caractéres sur un champ particulier, elles porteront la même étiquette puisqu'on parlera de la valeur
\end{attention}

\section{Analyse numérique}

\subsection{Analyse préliminaire}

\paragraph{}Une analyse numérique préliminaire donne les résultats suivants:

\begin{tabularx}{\textwidth}{|l|l|X|}
      \hline X & distinct(X) & maxfreq(X) \\ \hline
      epahttp & 1318 & 243  \\ \hline
      sdschttp & 75 & 12710 \\ \hline
      calgary & 4 & 16380  \\ \hline
\end{tabularx}

\paragraph{}Par la suite on supposera donc travailler sur des flux dont les éléments sortent d'un univers de taille 1318

\begin{tabularx}{\textwidth}{|l|l|X|}
      \hline X & Y & cod(X,Y) \\ \hline
      epahttp & sdschttp & -185 \\ \hline
      sdschttp & calgary & 177646 \\ \hline
      epahttp & calgary & -254  \\ \hline
\end{tabularx}

\subsection{Analyse sur le CM-sketch}

\paragraph{Evaluation sur 20 essais}Une première évaluation des résultats sur 20 évaluation CM-sketch sont regroupées dans le tableau ci-dessous 
(avec k = 16, $\delta$ = 0.0001 ou 0.0000001).\newline
On note que même si les résultats sont assez éloignés des valeurs réelles (ce qui était attendu), la ``topologie'',  l'allure de la situation est préservée (position relative des résultats)

\begin{tabularx}{\textwidth}{|l|l|X|X|}
      \hline X & Y & Moyenne CM\_sketch(X,Y, 16, 0.0001) 20 essais & Ecart type CM\_sketch(X,Y, 16, 0.0001) 20 essais \\ \hline
      epahttp & sdschttp & -248344 & 116404 \\ \hline
      sdschttp & calgary & 12311366 & 55424 \\ \hline
      epahttp & calgary & -369948  & 114896 \\ \hline
\end{tabularx}

\paragraph{Evaluation sur 100 mesures}Une évaluation plus fine a été effectuée avec 100 essais, k = 64. On notera déjà que les estimations sont bien plus 
proches de la réalité car k est plus proche de la taille de l'univers
(une évaluation en k= 1318 donne d'ailleurs les valeurs de codéviance exacte). Nous avons cherché à observer l'effet d'une diminution du $\delta$, 
censée augmenter la précision. Nous avons constaté une nette diminution de l'écart type.

\begin{tabularx}{\textwidth}{|l|l|X|X|X|}
      \hline X & Y & Moyenne CM\_sketch(X,Y, 64, 0.0001) 100 essais & Ecart type CM\_sketch(X,Y, 64, 0.0001) 100 essais & Ecart type CM\_sketch(X,Y, 64, 0.000001) 100 essais\\ \hline
      epahttp & sdschttp & -34199 & 17683 & 15939\\ \hline
      sdschttp & calgary & 3513236 & 670 & 325\\ \hline
      epahttp & calgary & -45931  & 22742 & 19680\\ \hline
\end{tabularx}

\begin{remarque}
 Une implémentation de l'algorithme distribué est proposée dans le code, bien qu'aucune application numérique n'en a été faite. 

\end{remarque}

\section{Conclusion}

En conclusion, l'algorithme d'approximation permet de minimiser le temps de calcul de la codéviance approchée, avec une influence des paramétres sur l'écart type confirmée par nos observations sur machine.

\end{document}